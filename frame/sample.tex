\begin{frame}[plain,t]
  {早川智一; 疋田輝雄. HTML5を用いた仮想Webブラウザの提案と評価. 情報処理学会論文誌, 2016.}
  %%%Image area
  \begin{minipage}[t]{\linewidth}
      \includegraphics[height=0.15\textheight,keepaspectratio]{img/img.png}
      \includegraphics[height=0.15\textheight,keepaspectratio]{img/img.png}
  \end{minipage}

  %%%Text area
  \begin{multicols}{2}
    \begin{beamercolorbox}[rounded=true, center, shadow=true,wd=\linewidth]{frametitle}
      どんなもの?
    \end{beamercolorbox}
    マルウェア感染メモリ領域を識別して,その領域からのシステムコールを検出する.
    
    \vfill
    \begin{beamercolorbox}[rounded=true, center, shadow=true,wd=\linewidth]{frametitle}
      先行研究と比べてどこがすごい?
    \end{beamercolorbox}
    プロセス全体からではなく,特定のメモリ領域ごとに検出が行える.

    \vfill
    \begin{beamercolorbox}[rounded=true, center, shadow=true,wd=\linewidth]{frametitle}
      技術や手法のキモはどこ?
    \end{beamercolorbox}
    \begin{enumerate}
      \item 改竄されない呼び出し解析(BTSの利用) \label{bts}
      \item メモリ確保情報をもとにした感染領域の判別 \label{infection}
    \end{enumerate}

    \newpage
    \begin{beamercolorbox}[rounded=true, center, shadow=true,wd=\linewidth]{frametitle}
      どうやって有効だと検証した?
    \end{beamercolorbox}
    \begin{itemlize}
      \item \ref{bts}: BTS Traceとスタックトレースの比較: 改竄に対しても有用であった
      \item \ref{infection}: MWSの動的活動観測 2014と同等のプログラムを検出(従来法ではできない)
    \end{itemize}

    \vfill
    \begin{beamercolorbox}[rounded=true, center, shadow=true,wd=\linewidth]{frametitle}
      議論はある?
    \end{beamercolorbox}
    直接実行に比べ $10\%$ 以下のパフォーマンススコア. 流石に遅くない? -> LBRの利用検討

    \vfill
    \begin{beamercolorbox}[rounded=true, center, shadow=true,wd=\linewidth]{frametitle}
      次に読むべき論文は?
    \end{beamercolorbox}
    Shinagawa, T. et al.: BitVisor: A Thin Hypervisor for Enforcing I/O Device Security(2009).

    VMMの理解のため.

  \end{multicols}
\end{frame}
