\begin{frame}[plain]
  {Digression: This presentation is created by Beamer}

  \begin{beamercolorbox}[rounded=true, center, shadow=true,wd=\linewidth]{frametitle}
    Beamer is A LaTeX class for producing presentations and slides
  \end{beamercolorbox}
  https://github.com/josephwright/beamer

  \begin{beamercolorbox}[rounded=true, center, shadow=true,wd=\linewidth]{frametitle}
    How install Beamer
  \end{beamercolorbox}
  If you have texlive, Beamer is already.

  \begin{beamercolorbox}[rounded=true, center, shadow=true,wd=\linewidth]{frametitle}
    My source codes
  \end{beamercolorbox}

  This presentation source place on
  
  https://github.com/forno/three\_min\_demo
\end{frame}

\begin{frame}[plain]
  {What is pytest}

  %%%Text area
  The pytest framework makes it easy to write small tests,
  yet scales to support complex functional testing for applications and libraries.

  \begin{beamercolorbox}[rounded=true, center, shadow=true,wd=\linewidth]{frametitle}
    {Features of pytest}
  \end{beamercolorbox}

  \begin{itemize}
    \item Assertion operators are just one. `assert`
    \item We can use some utility tools with decorator. `Fixture, Parametrize, TestGroups`
  \end{itemize}
\end{frame}

\begin{frame}[plain]
  {Demo with test-driven development}

  \begin{enumerate}
    \item Make one minimum test
    \item Make sure test failed
    \item Make minimum implementation (You can hardcode)
    \item Make sure \alert{ALL} test success
    \item Add second test
    \item Make sure test failed
    \item Update implementation (You can hardcode)
    \item Make sure \alert{ALL} test success
    \item Continue Add test, Make sure \structure{Test failed}, Update implemantation and Keep test success
    \item If you have no additional test, we get finish of implemantation
  \end{enumerate}
\end{frame}
