\begin{frame}[plain,t]
  {竹越智也; 萩原将文. 感情を付与した人型ロボット動作自動生成システム. 日本感性工学会論文誌, 2017.}
  %%%Image area
  \begin{minipage}[t]{\linewidth}
      \includegraphics[height=0.15\textheight,keepaspectratio]{img/img.png}
      \includegraphics[height=0.15\textheight,keepaspectratio]{img/img.png}
  \end{minipage}

  %%%Text area
  \begin{multicols}{2}
    \begin{beamercolorbox}[rounded=true, center, shadow=true,wd=\linewidth]{frametitle}
      どんなもの?
    \end{beamercolorbox}
    基本動作をベクトル化し,ベクトル変換を行う事で基本動作に感情を付与する
    
    \vfill
    \begin{beamercolorbox}[rounded=true, center, shadow=true,wd=\linewidth]{frametitle}
      先行研究と比べてどこがすごい?
    \end{beamercolorbox}
    生成された動作に感情を付与するところ

    \vfill
    \begin{beamercolorbox}[rounded=true, center, shadow=true,wd=\linewidth]{frametitle}
      技術や手法のキモはどこ?
    \end{beamercolorbox}
    入力動作→ベクトル→低次元化(26次元(頭2両腕両足6)から4次元に)+感情→動作ベクトル→出力動作

    \newpage
    \begin{beamercolorbox}[rounded=true, center, shadow=true,wd=\linewidth]{frametitle}
      どうやって有効だと検証した?
    \end{beamercolorbox}
    予備実験で特徴量を決定,提案手法で生成した動作を被験者に見せて被験者実験。正答率60\%,PCAだと4感情→2感情にすると80\%に。感情ごとの動作の特徴の分析

    \vfill
    \begin{beamercolorbox}[rounded=true, center, shadow=true,wd=\linewidth]{frametitle}
      議論はある?
    \end{beamercolorbox}
    正答率の低さは学習に使うデータの少なさが原因→誤差計算してなくね?

    \vfill
    \begin{beamercolorbox}[rounded=true, center, shadow=true,wd=\linewidth]{frametitle}
      次に読むべき論文は?
    \end{beamercolorbox}
    竹越智也,萩原将文:ロボット漫才自動生成システム,日本感性工学会.15(1),pp.47-54,2016


  \end{multicols}
\end{frame}
